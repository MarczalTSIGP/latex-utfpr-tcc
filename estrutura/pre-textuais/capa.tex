% CAPA---------------------------------------------------------------------------------------------------

% ORIENTAÇÕES GERAIS-------------------------------------------------------------------------------------
% Caso algum dos campos não se aplique ao seu trabalho, como por exemplo,
% se não houve coorientador, apenas deixe vazio.
% Exemplos:
% \coorientador{}
% \departamento{}

% DADOS DO TRABALHO--------------------------------------------------------------------------------------
\titulo{Título do Trabalho: Subtítulo do Trabalho}
\titleabstract{Title in English}
\autor{Nome Completo do Autor}
\autorcitacao{SOBRENOME, Nome} % Sobrenome em maiúsculo
\local{Guarapuava}
\data{2016}

% NATUREZA DO TRABALHO-----------------------------------------------------------------------------------
% Opções:
% - Projeto de Trabalho de Conclusão de Curso de graduação
% - Monografia de Trabalho de Conclusão de Curso (se for Graduação)
% - Dissertação (se for Mestrado)
% - Tese (se for Doutorado)
% - Projeto de Qualificação (se for Mestrado ou Doutorado)
\projeto{Monografia de Trabalho de Conclusão de Curso de graduação}

% TÍTULO ACADÊMICO---------------------------------------------------------------------------------------
% Opções:
% - Bacharel ou Tecnólogo (Se a natureza for Trabalho de Conclusão de Curso)
% - Mestre (Se a natureza for Dissertação)
% - Doutor (Se a natureza for Tese)
% - Mestre ou Doutor (Se a natureza for Projeto de Qualificação)
\tituloAcademico{Tecnólogo em Sistemas para Internet}

% ÁREA DE CONCENTRAÇÃO E LINHA DE PESQUISA---------------------------------------------------------------
% Se a natureza for Trabalho de Conclusão de Curso, deixe ambos os campos vazios
% Se for programa de Pós-graduação, indique a área de concentração e a linha de pesquisa
\areaconcentracao{}
\linhapesquisa{}

% DADOS DA INSTITUIÇÃO-----------------------------------------------------------------------------------
% Se a natureza for Trabalho de Conclusão de Curso, coloque o nome do curso de graduação em "programa"
% Formato para o logo da Instituição: \logoinstituicao{<escala>}{<caminho/nome do arquivo>}
\instituicao{Universidade Tecnológica Federal do Paraná}
\departamento{} % Deixa em branco caso não exista departamento
\programa{Curso de Tecnologia em Sistemas para Internet}
\logoinstituicao{0.2}{dados/figuras/logo-instituicao.png}

% DADOS DOS ORIENTADORES---------------------------------------------------------------------------------
\orientador{Nome do orientador}
%\orientador[Orientadora:]{Nome da orientadora}
\instOrientador{Instituição do orientador}

\coorientador{Nome do coorientador}
%\coorientador[Coorientadora:]{Nome da coorientadora}
\instCoorientador{Instituição do coorientador}

% Quando existir mais de um coorientador-----------------------------------------------------------------
% \coorientador[Coorientadores:]{Prof. Me. XXXXXX \newline
% Universidade Tecnológica Federal do Paraná - Câmpus Guarapuava \newline
% \newline Prof. Dr. XXXXX. \newline Universidade Tecnológica
% Federal do Paraná - Câmpus Guarapuava}
